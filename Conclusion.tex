\chapter*{Conclusion}
\addcontentsline{toc}{chapter}{Conclusion}
This thesis presented contributions to the topics of camera calibration, stereo shape estimation, and surface reconstruction for the application of reconstructing 3\hyp D models of underwater cave systems. In order to help calibrate camera systems with extreme distortion such as the GoPro in SuperView mode, we studied calibration techniques that took into account the removal of outlier images. MATLAB's computer vision system toolbox was found the be the most effective way to handle the problems of high distortion systems, and a simplified OpenCV version of their outlier removal was ported over. With a successfully calibrated camera we presented the first ever reconstruction results from an underwater cave using a novel approach utilizing the artificial lighting of the scene as a tool to map the boundaries. This approach was applied on actual collected cave dive footage, and was effective in both producing a reconstruction and not interfering with the standard procedure the cave divers follow. Lastly the point cloud underwent a series of tests for surface reconstruction using a number of the most prominent techniques in the field.

We identified a number of areas for which future work can expand upon through the study of our tests and results. Different camera systems with a larger baseline and even more reduced calibration error will hopefully create more accurate reconstructions that in turn lead to more accurate surfaces. The inclusion of additional non obtrusive sensors such as sonar can also increase the accuracy of our results. 

Cave mapping is a problem with significant impact across a wide range of fields. Steps made towards increasing the accuracy and efficiency of these mappings, especially at a low cost and non-cumbersome level, is vital. This work introduces a pivotal first step towards achieving complete mappings.. 