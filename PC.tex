\chapter{Stereo Shape Estimation}\label{chap:pcrecon}
\section{Overview}\label{sec:pcoverview}
Camera calibration was an important step in our overall pipeline of underwater cave reconstruction, because without it we could not accurately project 2\hyp D image points into 3\hyp D space. The footage for all of our tests were taken using a GoPro in SuperView mode, so this calibration problem needed to be addressed. With calibration out of the way, we now shift focus to the generation of a geometrically accurate point cloud using stereo footage collected in an underwater cave. The presented approach utilizes the presence of the artificial lighting to produce a rough model of the traversed area. In particular, the video\hyp light cone is used to identify the  walls of the cave from a single stereo pair. Furthermore, motion between consecutive stereo pairs is estimated and the 3\hyp D reconstruction is utilized to produce an approximate volumetric map of the cave.

\section{Related Work}\label{sec:pcrelwork}
The majority of underwater mapping up to now consists of fly\hyp overs with downward pointing sensors mapping the floor surface. The resulting representation consists of 2.5 dimensional mesh\hyp maps or image mosaics with minimal structure in the third dimension. In addition to underwater caves,  several other underwater environments  exhibit prominent three dimensional structure. Shipwrecks, are significant historical sites.  Producing accurate photorealistic 3\hyp D  models of these wrecks will assist in historical studies and also monitor their deterioration over time. During maritime disasters, it is important to produce accurate maps of the sunken vessel, especially the interior, in order to assist with rescue efforts. Multiple cases existed where survivors have been rescued from submerged vessels~\cite{Gray2003} hours, or even days after the event. Finally, underwater infrastructure inspection~\cite{Ribas2008} is another dangerous and tedious task that is required to be performed at regular intervals. Such infrastructure includes bridges, hydroelectric dams~\cite{Ridao2010}, water supply systems~\cite{White2010}, and oil rigs. For more information please refer to the Massot\hyp Campos and Oliver\hyp Codina survey~\cite{massot2015optical} for an overview of 3\hyp D sensing underwater. 

Most of the underwater navigation algorithms~\cite{lee2005underwater,snyder2010doppler,johannsson2010imaging,rigby2006towards} are based on acoustic sensors such as Doppler Velocity Log (DVL), ultra-short baseline (USBL) and sonar.  Gary et. al.~\cite{gary20083d} presented a 3D model of a cenote using LIDAR and sonar data collected by DEPTHX (DEep Phreatic THermal eXplorer) vehicle having DVL, IMU and depth sensor for underwater navigation. Corke et. al.~\cite{corke2007experiments} compared acoustic and visual methods for underwater localization. However, collecting data using DVL, sonar, and USBL while diving is expensive and sometimes not suitable in cave environments. 

Using stereo vision underwater has been proposed by several groups, however, most of the work has focused on open areas with natural lighting, or artificial light that completely illuminates the field of view. Small area dense reconstruction of a lit area was  proposed by Brandou et al.~\cite{Brandou2007}. Mahon et. al.~\cite{mahon2008efficient} proposed a SLAM algorithm based on the viewpoint augmented navigation (VAN) using stereo vision and DVL in underwater environment. A framework proposed by Leone et al.~\cite{Leone2008} operated over mainly flat surfaces. Several research groups have investigated the mapping and/or inspection of a ship's hull using different techniques~\cite{Hogue2007,Hover2007,Englot2013,akim-2013a}, the most famous shipwreck visual survey being that of the Titanic~\cite{Eustice2006}. Error analysis was performed recently by Sedlazeck and Koch \cite{Sedlazeck2012}. The problem of varying illumination was addressed by Nalpantidis et al.~\cite{Nalpantidis2010} for above\hyp ground scenes in stereo reconstruction. More recently, Servos, Smart and Waslander~\cite{Servos2013} presented a stereo SLAM algorithm with refraction correction in order to address the transitions between water, plastic, and air that exist in the underwater domain.