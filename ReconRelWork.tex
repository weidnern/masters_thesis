\section{Related Work}\label{sec:reconrelwork}
Surface reconstruction is a problem that expands well beyond scene reconstruction from a camera. Many methods have developed for data collected with range scan datasets. Luckily these methods can often be applied generically to any kind of data set, but there are key components and limiting factors that make some better than others. Such data sets as those collected with underwater stereo images often fail under the constraints of popularized techniques. 

A subcategory of surface reconstruction techniques is that of point interpolation. In this category, the input data points would all represent points exactly on the surface of the reconstruction object. Because this is  in practice impossible without at least some minimal error, applications take measurement error into account, but assume it to be small. These methods are thus popular on range scan data sets where the volume of the data points is high, the surface is usually smooth, and the degree of error is often small. Popular methods include the cocone method~\cite{amenta2000simple} and the power crust method~\cite{amenta2001power} which use a piecewise linear approximation of the surface based on a Delaunay triangulation of the input points. Alternatively there is the ball-pivoting algorithm~\cite{817351}  which incrementally builds a surface based on the idea of a ball rolling across and intersecting points. These methods are also incredibly susceptible to noise due to their assumption that the points exactly lie on the object surface. This can mitigated by noise smoothing or popular point sampling techniques such as Poisson disk sampling~\cite{corsini2012efficient}, but the loss of data can result in an output to abstract from the true reconstruction. 

Alternative to point interpolation, there is also the category of algorithms that deal with point approximation. This removes the restriction that points must lie on the on the reconstruction object, and the methods are much more varied. Many of these techniques require knowledge of the point normals, which is absent information when generating input points from images. Thus methods have developed for determining and orienting the normals of points based on estimating the normal of a plane tangent to the surface~\cite{RusuDoctoralDissertation}. Known point normal information now introduces the use of approximation reconstruction techniques. Signed distance function approaches such as those developed by Hoppe et al.~\cite{Hoppe:1992:SRU:133994.134011,Hoppe:1995:SRU:221616} define a surface in implicit form and then allow for the extraction of a final triangulated mesh using popular approaches such as marching cubes~\cite{Lorensen:1987:MCH:37401.37422}. Instead of using a distance function, methods such as Poisson reconstruction~\cite{Kazhdan:2006:PSR:1281957.1281965} use indicator functions to define a water tight model. Lastly there are approaches using moving least-squares(MLS) which were introduced by Levin~\cite{levin2004mesh} and soon popularized through point set surfaces by Alexa et al.~\cite{alexa2003computing} All together these form a vast library of surface reconstruction techniques to use between the interpolation and approximation subcategories. 

With such a large volume of available solutions to the surface reconstruction problem, it is important to identify the best approach given a specific data set. In our case we need to identify a method that can deal with the sparsity of our point cloud volume while also not falling susceptible to noise. Other works have done similar data collection and reconstruction of underwater systems using cameras as their primary means of data collection. Johnson-Roberson et al.~\cite{johnson2010generation} used Delaunay triangulation to reconstruct individual top down stereo images of the sea floor. These individual reconstructions are then aggregated together using a Volumetric Range Image Processing (VRIP) technique developed by Curless and Levoy. Campos et al.~\cite{campos2014surface} introduced a method for 3D surface reconstruction with an emphasis on underwater optical mapping. This method is inspired by restricted Delaunay triangulation and is robust to noise and outliers often attributed with underwater data collection. Unfortunately these two methods have clear advantages. The former only scans directly below at the more or less flat ocean floor, while the latter builds a point set from a large collection of camera angles and locations. Another study by Tischenko~\cite{tishchenko_2010} used a range scan and collected data out of water, but compared the best approaches to reconstructing a hollow cylindrical point cloud shape. This is similar to the shape of our cave system and can thus prove useful. This study found the marching cubes implementation mentioned earlier by Hoppe to produce the most accurate reconstruction results. 


%Solutions to surface reconstruction from point clouds vary depending on the domain and the input data. Factors such as the density of the cloud, the kind of preliminary structural data coinciding with the cloud, and the noise of the cloud all define the approach used in reconstruction. Our method for point cloud generation creates a cylindrical shape for the cave at both a sparse and dense level depending on the number of input frames used. In both cases the cloud contains noise and lacks any predefined data on the values for the point normals. Previous work by Tishchenko ~\cite{tishchenko_2010} compares the advantages of certain surface reconstruction algorithms, and while the input clouds were generated above water and using a laser, they share the properties of forming a cylindrical shape without point normals and noise present. These comparisons highlight a method for calculating point normals based on point neighborhoods and distinguish a Marching Cubes surface reconstruction method by Hoppe as the appropriate approach to a problem with these parameters~\cite{Hoppe:1992:SRU:133994.134011,Hoppe:1995:SRU:221616}. Other techniques such as Poisson surface reconstruction and Delaunay triangulation are used heavily in the underwater domain ~\cite{campos2014surface}. 
%
%A powerful tool for implementing surface reconstruction techniques is the open source application Meshlab. Meshlab includes implementation of normal estimation, marching cubes ~\cite{guennebaud2008dynamic}, and Poisson reconstruction ~\cite{Kazhdan:2006:PSR:1281957.1281965}. It also includes an implementation of the Ball-Pivoting Algorithm ~\cite{mittlemanball} that produces interesting results, while still succumbing to the drawbacks highlighted in Tishchenko's comparison. Lastly, Meshlab also has an implementation of Poisson-disk sampling ~\cite{corsini2012efficient}. Sampling of point cloud sets allows reconstruction algorithms to avoid noise errors and unnecessary over fitting, but the dsrawback is a loss of accuracy from the volume of input points. Sampling has a varying effect on the output depending on the algorithm, and in cases of Ball-Pivoting the results are highly dependent on it. A challenge for all of these algorithms is that they rely on a set of input parameters that can vary the output. Finding the proper parameters introduces another level of complexity to outputting an accurate final mesh.